
\section{Conclusion}
This analysis shows that it is possible to select a sample with a significant component of $\nu_{e}$ CC0$\pi$-Np events in a mainly $\nu_\mu$ beam. An optical selection stage, followed by a topology requirement and a series of kinematic and geometric cuts is able to achieve a $(14.3\pm0.2)\%$ efficiency and a $(24.0\pm0.3)\%$ purity. This corresponds to 4.9 $\nu_{e}$ CC0$\pi$-Np events for a MicroBooNE exposure of \num{4.84e19} POT. The total number of selected events is 21 in the data and 20.4 in the Monte Carlo simulation. 
The reconstructed energy spectrum of the selected events has been measured using a calorimetric-based technique for the shower-like objects and a length-based technique for the track-like objects. The agreement between the data and Monte Carlo reconstructed energy spectra is also good, with $\chi^{2} / \mathrm{n.d.f.} = 0.51$.
The selection has been validated checking the agreement between data and Monte Carlo also in a photon-enhanced sample and in a $\nu_\mu$-enhanced sample.
Expected improvements in the reconstruction of the objects in the TPC, in the background rejection and in the results of the complementary analyses will allow us to further improve the significance of the $\nu_{e}$ CC0$\pi$-Np component, necessary to confirm or rule out the MiniBooNE low-energy excess.