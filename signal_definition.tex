% In this section, highlight the following:
% Why are we picking to do 1eNp?
% - Similar to Miniboone, in that this focuses on events from a predominantly CCQE interaction
% - Addition of vertex activity (Np) allows more aggressive cosmic rejection, and improves vertex tagging in Pandora
% - Requirement of only protons at the vertex reduces statistics, but allows for a track-length based calorimetry of protons which improves resolution.

\section{Signal definition}
The Standard Model, at tree level, allows only weak interaction for neutrinos. As such, neutrinos exchange a $W^{\pm}$ vector boson in charged-current (CC) interactions and a $Z$ vector boson in neutral-current (NC) interactions. 
However, the energy of a neutrino can range from few keV to several PeV, as recently measured by the IceCube experiment \cite{icecube}, producing wildly different topologies in the final state. 
In our case, we are interested in $\nu_{e}$ CC interactions in the sub-GeV region (0.1 - 1~GeV). In a naive scenario, where we ignore nuclear and final-state interactions, we can have three dominant outcomes:
\begin{itemize}
\item Charged Current Quasi Elastic (CCQE): it is the principal signature for most neutrino experiments. The neutrino exchanges a $W^{\pm}$ with a neutron, producing a proton and a charged lepton in the final state.
\item Charged Current Resonant production (CCRES): the neutrino excites a nucleon, which emits a pion.
\item Charged Current Deep Inelastic Scattering (CCDIS): the neutrino interacts with a quark, producing a shower of particles as the nucleus breaks apart.
\end{itemize}

However, hadrons exiting the nucleus after the neutrino interaction can re-interact and change identity or eject other hadrons (Final State Interactions, FSI) \cite{ccqe2}. It is then necessary to define the signal by the particles in its final state. Our selection aims to have a sample with one electron, no other leptons or photons, at least one proton, and no other hadrons or mesons. This kind of event is called CC0$\pi$-Np (where N > 0) \cite{teppei}.

In a LArTPC, the final state of a contained $\nu_{e}$ CC0$\pi$-Np interaction will correspond in general to one or more ionisation tracks, produced by the protons, and an electromagnetic shower, produced by the electron. A charged pion in the final state, instead, will decay mainly into a muon, producing a second ionisation track, while a neutral pion will decay into two photons, producing two electromagnetic showers. 
CCDIS interactions, whose branching ratio is relatively small in the sub-Gev region, usually produce large hadronic jets, which are reconstructed as a combination of tracks and showers. 

%As such, a perfect reconstruction of a $\nu_{e}$ CCQE-like event in a LArTPC will produce as many ionisation tracks as the number of protons in the final state and a single electromagnetic shower (the electron), sharing a common vertex.

The MiniBooNE experiment showed an excess of CCQE-like events (one proton and one electron in the final state) in the 200-475~MeV neutrino energy range \cite{miniboone}, so our analysis will focus on a similar topology in an analogous energy range.